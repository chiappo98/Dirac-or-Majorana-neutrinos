\documentclass{subnucbo}
\usepackage[utf8]{inputenc}
\usepackage{graphicx}
\usepackage{amsmath,amssymb}
\usepackage{float}
\usepackage{siunitx}
\usepackage{booktabs}

\begin{document}

\title{Dirac or Majorana neutrinos: an open question}

\author{F. Chiapponi} 

\instlist{\inst  Dipartimento di Fisica Universit\`a di Bologna, Via Irnerio 46 - 40126 Bologna, Italy} 

\acyear{2020--2021}

\maketitle

\begin{abstract}
In this paper I will present the state of the research on neutrinos, together with several still open questions. In particular our focus will be set on wether neutrinos are Dirac or Majorana particles. In the end we will describes the experiments that attempt to observe a neutrinoless double beta decay in even-even nuclei%~\ref{}
, which will prove the Majorana nature of these particles.
\end{abstract}

\section{An elusive particle}
\subsection{History}
The neutrino was first theorized in 1930 by Wolfgang Pauli in order to account for conservation of energy and momentum in beta decays.
At the beginning these processes were though to proceed as
\[
^A_Z X\rightarrow ^A_{Z+1}Y+e^-
\]
However looking at the energy spectrum of emitted electrons (\emph{Fig.}~\ref{fig:betaD}) it was clear that it was due to a three body process, but then a particle was missed. For a two body decay the electrons' energy spectrum should form a sharp peak around 
\begin{equation}
E=\biggl(\frac{m^2_A-m^2_B+m^2_e}{2m_A}\biggr)c^2
\label{eq:Emax}
\end{equation}
This energy corresponds only to the maximum electron energy of \emph{Figure}~\ref{fig:betaD}.

\begin{figure}[h]
\centering
\includegraphics[width=0.6\textwidth]{energy_spectrum.png}
\caption{The beta decay spectrum of tritium ($^3_1H\rightarrow ^3_2He$). (\emph{Source}: G.M. Lewis, \emph{Neutrinos} (London: Wykeham, 1970))}
\label{fig:betaD}
\end{figure}

The problem of the missing energy could be solved introducing a new particles, which had to be neutral to conserve electric charge and extremely light, from the fact that the observed electron enrgies range up to the value given in equation ~\ref{eq:Emax}.
Fermi called this particle \emph{neutrino}, and it was finally discovered in 1956 by Cowan and Reines at the Savannah River nuclear reactor in South Carolina, where they set up a large tank of water and watched for the inverse beta decay reaction.%see later/section ~\ref{sec:}

Today we know the existance of three flavours of neutrinos, associated to the correspondig charged leptons %~\ref{} (maybe inside caption)
:
\begin{table}[H]
\begin{tabular}{cccc}
\toprule
Lepton & Charge & Mean life (s) & Mass ($MeV/c^2$)\\
\midrule
$e^-$ & $-e$ & $\infty$ & 0.511\\
$\nu_e$ & 0 &  & 0\\
$\mu^-$ & $-e$ & $2.197\cdot10^{-6}$ & 105.658\\
$\nu_\mu$ & 0 & & 0\\
$\tau^-$ & $-e$ & $(291.0\pm1.5)\cdot10^{-15}$ & 1777\\
$\nu_\tau$ & 0 &  & 0\\
\bottomrule
\end{tabular}
\caption{Leptons.}
\label{tab:classification}
\end{table}
 
According to the Standard Model (SM) neutrinos are \emph{massless} fermions. Leptons in the table above all hav Total Lepton Number $L=+1$, while their antiparticles have $L=-1$, toghether with opposite electric charge.

\subsection{Neutrinos production and interaction}

\begin{eqnletter}
\pi^+ &\rightarrow& \mu^++\nu_\mu \\
\mu^- &\rightarrow& e^-+\bar{\nu}_e+\nu_\mu
\end{eqnletter}

 Being neutral leptons they interact only through weak interactions, and hence are very difficult to detect.


\subsection{Fermi Theory}%?????
\subsection{Dirac equation}
\subsection{Majorana neutrinos}

\section{Nutrinoless double beta decay}




\subsection{Mathematics}
Here is a lettered array~(\ref{e.all}), with eqs.~(\ref{e.house})
and~(\ref{e.phi}):
\begin{eqnletter}
 \label{e.all}
 \drm x_\sy{F} & = & 1.2\cdot10^3\un{cm}, \qquad
                     \tx{where\ } \sy{F} = \tx{Fermi}    \label{e.house}\\
 \phi_i        & = & i\pi                                \label{e.phi}
\end{eqnletter}

\subsection{Tables}

Tables~\ref{tab:pricesI}
inserted at this point.yes, but...

\begin{table}[]
  \caption{Prices of important items.}
  \label{tab:pricesI}
  \begin{tabular}{lcr}
  \hline
  Ice-cream      & 2.50  & euro    \\
  More ice-cream & 2500 & euro    \\
  Crocodile      & 1500  & dollars \\
  \hline
  Phone call     & .25   & dollars \\
  X-Men          & 1.25  & dollars \\
  Dollar         & 1     & dollars \\
  \hline 
  \end{tabular}
\end{table}

\subsection{Citations}
We're almost done, just some citations~\cite{ref:apo}
and we will be over~\cite{ref:pul,ref:bra}.


\appendix

\section{}
Let us go then, you and I\ldots

\acknowledgments
This work was produced, supported and perpetrated by M. Bellacosa under
the auspices of the Italian Physical Society.

\begin{thebibliography}{0}
\bibitem{ref:apo} \BY{Boccaccio~G. \atque de~Cam\~oes~L.}
  \IN{Phys. Rev. A}{13}{1999}{12};
  \SAME{69}{999}{1666}.
\bibitem{ref:pul} \BY{Pulci~L.}
  preprint INFN 8181.
\bibitem{ref:bra} \BY{Bragg~B.}
  \TITLE{Tender comrade},
  in \TITLE{Workers Playtime},
                  edited by \NAME{Tizio A. \atque Caio B.}
                  (Unexeditor, Bologna) 1997, pp.~1-10.
\end{thebibliography}

\end{document}