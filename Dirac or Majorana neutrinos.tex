\documentclass{subnucbo}
\usepackage[utf8]{inputenc}
\usepackage[english]{babel}
\usepackage{graphicx}
\usepackage{amsmath,amssymb}
\usepackage{float}
\usepackage{siunitx}
\usepackage{booktabs}
\usepackage{mathrsfs}
\usepackage{siunitx}
\usepackage{wrapfig}
\usepackage{mhchem}

\begin{document}

\title{Dirac or Majorana neutrinos: an open question}

\author{F. Chiapponi} 

\instlist{\inst  Dipartimento di Fisica Universit\`a di Bologna, Via Irnerio 46 - 40126 Bologna, Italy} 

\acyear{2020--2021}

\maketitle

\begin{abstract}
In this paper I will present the current state of research on neutrinos and in particular I will focus on the open question whether neutrinos are Majorana or Dirac particles i.e. if they correspond to their antiparticles or not. The theories which try to account for these two possibilities will be discussed together with the experiments that attempt to answer this question~\cite{ref:front,ref:cupid}. Indeed the observation of the neutrinoless double beta decay in even-even nuclei would assert that neutrinos are Majorana particles. 
\end{abstract}

\section{An elusive particle}
\subsection{History}
The neutrino was first theorized in 1930 by Wolfgang Pauli in order to account for conservation of energy and momentum in beta decay.
At the beginning this process was though to proceed as
\[
^A_Z X\rightarrow ^A_{Z+1}Y+e^-
\]
However looking at the energy spectrum of emitted electrons (Fig.~\ref{fig:betaD}) it was clear that it was due to a three body process, but then a particle was missed. For a two body decay the electrons' energy spectrum should form a sharp peak around 
\begin{equation}
E=\biggl(\frac{m^2_A-m^2_B+m^2_e}{2m_A}\biggr)c^2
\label{eq:Emax}
\end{equation}
This energy corresponds only to the maximum electron energy of Figure~\ref{fig:betaD}.

\begin{figure}[h]
\centering
\includegraphics[width=0.6\textwidth]{energy_spectrum.png}
\caption{The beta decay spectrum of tritium ($^3_1H\rightarrow ^3_2He$). \emph{Source}: G.M. Lewis, \emph{Neutrinos} (London: Wykeham, 1970)}
\label{fig:betaD}
\end{figure}

The problem of the missing energy could be solved introducing a new particles, which had to be neutral to conserve electric charge and extremely light, from the fact that the observed electron energies range up to the value given in equation (\ref{eq:Emax}).
Fermi called this particle \emph{neutrino}, which was finally discovered in 1956 by Cowan and Reines at the Savannah River nuclear reactor in South Carolina, where they set up a large tank of water and watched for inverse beta decay reactions (see section \ref{sec:prod}).

Today we know the existence of three flavors of neutrinos, associated to the corresponding charged leptons, as shown in Table~\ref{tab:classification}~\cite{ref:cott}.
\begin{table}[H]
\begin{tabular}{cccc}
\toprule
Lepton & Charge & Mean life (s) & Mass (MeV)\\
\midrule
$e^-$ & $-e$ & $\infty$ & 0.511\\
$\nu_e$ & 0 &  & 0\\
$\mu^-$ & $-e$ & $2.197\cdot10^{-6}$ & 105.658\\
$\nu_\mu$ & 0 & & 0\\
$\tau^-$ & $-e$ & $(291.0\pm1.5)\cdot10^{-15}$ & 1777\\
$\nu_\tau$ & 0 &  & 0\\
\bottomrule
\end{tabular}
\caption{Leptons.}
\label{tab:classification}
\end{table}
 
According to the Standard Model (SM) neutrinos are \emph{massless} fermions. Leptons in the table above all have Total Lepton Number $L=+1$, while their antiparticles have $L=-1$, together with opposite electric charge.

An interesting feature of weak interactions is that it does not conserve parity. Indeed, it was demonstrated by Madame Wu in 1957 that only left-handed component of electrons and neutrinos participate in the interaction, and the same for other flavors. Conversely antiparticles enter in interactions with right-handed component.

\subsection{Neutrinos production and interaction}
\label{sec:prod}
Neutrinos are produced in leptonic (\ref{eq:Lep}) and semileptonic decays (\ref{eq:semiL1} - \ref{eq:semiL3}) of particles involving weak interactions, but also in weak decays of unstable atomic nuclei. 
\begin{eqnletter}
\label{eq:lep}
\mu^- &\rightarrow& e^-+\bar{\nu}_e+\nu_\mu \label{eq:Lep}\\
\pi^+ &\rightarrow& \mu^++\nu_\mu \label{eq:semiL1}\\
K^+ &\rightarrow& \pi^0+\mu^++\nu_\mu \label{eq:semiL2}\\
&\hookrightarrow& \pi^0+e^++\bar{\nu}_e \label{eq:semiL3}
\end{eqnletter}
On Earth the flux of neutrinos is dominated by Solar neutrinos, produced in fusion processes inside the Sun, and neutrinos of atmospheric origin, due to the interaction of cosmic rays with the atmosphere. However they are copiously produced also in nuclear power plants, which often are used as neutrinos source for particle physics experiments.

Weak processes can be classified in charged current (CC) and neutral current (NC) interactions, whether or not there is a change in the electric charge between initial and final fermions. CC interactions are mediated by the charged $W^-$ and $W^+$ bosons, while NC interaction by the $Z$ boson, which is neutral.
These mediators are massive \cite{ref:pdgBos}:
\begin{gather}
M_W=\SI{80.379\pm0.012}{GeV} \\
M_Z=\SI{91.1876\pm0.0021}{GeV}
\end{gather}
It is this fact that makes interactions mediated by these bosons weak and short-ranged.%insert weak coupling and cross sections?

Being neutral leptons neutrinos interact only through weak interactions, and hence are very difficult to detect. 
On the other side they are very useful to study weak interactions, since the weak transition amplitude of processes in which they are involved is not overcame by strong or electromagnetic amplitudes.
Neutrinos can be detected through the inverse beta decay interaction
\begin{equation}
\bar{\nu}_e+p \rightarrow e^++n
\end {equation}
that is, observing the emitted charged lepton.
Another interaction channel is elastic scattering 
\begin{equation}
\nu_i+e^- \rightarrow \nu_i+e^- \quad \text{where $i=e, \mu, \tau$}
\end{equation}
which has the advantage of involving all neutrino flavors.

\subsection{Neutrino mixing and CP violation}
Several experiments studied neutrino fluxes coming from different sources, such as nuclear reactors, the Sun and the atmosphere. They observed some discrepancies:
\begin{itemize}
\item The flux of electron neutrinos arriving on Earth form the Sun is about an half of that expected from theory (the so called Solar Neutrino Problem).
\item The observed ratio of $\nu_{\mu}/\nu_e$ events from atmospheric neutrinos is only $\sim0.6$ of the predicted value. This deficit is known as the Atmospheric Neutrino Anomaly.
\end{itemize}
Neutrino oscillations i.e. electron neutrinos turning into muon and tau neutrinos, are able to explain these experimental data \cite{ref:wrap}. The most important consequence of the discovery of neutrino mixing is that $e$, $\mu$ and $\tau$ neutrinos cannot be massless, as conceived by Bruno Pontecorvo in 1968. 
%say something on Majorana neutrinos??

\section{Theory}

\subsection{Dirac equation}
The Dirac equation was invented by P.A.M. Dirac in seeking to make Schrödinger's equation for an electron compatible with special relativity. The Dirac field $\psi$ is a four component spinor, which can describe both particles and antiparticles, with half-integer spin.
The Dirac equation is
\begin{equation}
\label{eq:dir}
(i\gamma^{\mu}\partial_{\mu}-m)\psi=0
\end{equation}
where the Dirac spinor is composed by a left-handed spinor and a right-handed spinor. %è sempre così o è una particolare base?
\begin{equation} 
\psi=\begin{pmatrix}\psi_1\\ \psi_2 \\ \psi_3 \\ \psi_4 \end{pmatrix}=\begin{pmatrix}\psi_L\\ \psi_R \end{pmatrix}
\end{equation}

From Dirac equation it is possible to build a Lagrangian density
\begin{equation}
\label{eq:DirLag}
\mathcal{L}=\bar{\psi}(i\gamma^{\mu}\partial_{\mu}-m)\psi
\end{equation}
where $\bar{\psi}$ is the row matrix $\bar{\psi}=\psi^{\dagger}\gamma^0$.

Equation (\ref{eq:dir}) has plane wave solutions of the form
\begin{equation}
\psi_L=u_Le^{i(\vec{p}\cdot\vec{r}-Et)},
\qquad
\psi_R=u_Re^{i(\vec{p}\cdot\vec{r}-Et)}
\end{equation}
where $u_L$ and $u_R$ are two component spinors. For simplicity lets choose the particle rest frame, i.e. in which $\vec{p}=0$. Since Dirac equation must satisfy special relativity we have that $E^2=p^2+m^2$ and therefore $E=\pm m$. We thus find positive and negative energy solutions%solve problem about spinors u and u_L
\begin{gather}
\psi_L=ue^{-imt},\qquad  \psi_R=ue^{-imt} \\
\psi_L=ve^{imt},\qquad  \psi_R=-ve^{imt}
\end{gather}
The latter have to be interpreted as positive energy antiparticle states.
The total Lagrangian density which describes weak interactions of leptons is determined by the requirement of $U_Y(1)\times SU_L(2)$ local gauge symmetry.

\subsection{Dirac neutrinos}
Generation of charged lepton masses is due to spontaneous symmetry breaking (SSB) after the coupling with the Higgs field.
This mechanism could in principle be applied also to neutrinos, however since the Standard Model does not contain right-handed neutrinos, no such Yukawa interaction can be built for neutrinos, which are consequently massless.

In order to generate neutrino masses then we have to go beyond the Standard Model, adding an arbitrary number of sterile neutrinos.
One possible solution is to introduce three new sterile (right-handed) neutrinos so that neutrinos acquire mass through the usual Yukawa mechanism.  However, since values of neutrino masses are many order of magnitude smaller than those of other fermions it is very unlikely they have a common SM Higgs origin.
Another possibility is the method of the effective Lagrangian which, after SSB generates Majorana neutrinos with definite masses and suppression via seesaw mechanism of neutrino masses with respect to SM masses of leptons and quarks.

If neutrinos are Dirac particles total lepton number \emph{L} is conserved, and neutrino $\nu_i$ and antineutrino $\bar{\nu}_i$ have lepton number respectively  $L(\nu_i)=+1$ and $L(\bar{\nu}_i)=-1$.
For Majorana neutrinos instead L is not conserved and it's not possible to distinguish neutrinos from antineutrinos through any quantum number.

\subsubsection{Higgs mechanism for fermion masses}
Generation of fermion masses through the standard Higgs mechanism requires the introduction of a Yukawa-type coupling of the lepton SU(2) doublet to a scalar field, which then acquires a vacuum expectation value (vev) $v$: 
\begin{equation}
\label{eq:ElectronLag}
\mathcal{L}_Y^e(x)=-g_e(\bar{\psi}_{eL}\phi e_R+\bar{e}_R\phi \psi_{eL}) 
\end{equation}
where
\begin{equation}
\psi_{eL}(x)=\begin{pmatrix}\nu_{e}(x)\\ e^-(x) \end{pmatrix}_L,\qquad
\phi=\begin{pmatrix}\phi_+(x)\\ \phi_0(x) \end{pmatrix}
\end{equation}
are lepton and Higgs doublets, while $e_R(x)$ is the right handed electron singlet field.
We spontaneously break the symmetry choosing 
\begin{equation}
\phi(x)=\begin{pmatrix}0\\ \frac{v+H(x)}{\sqrt{2}}\end{pmatrix}
\end{equation}
with $v=(\sqrt{2}G_F)^{-1/2}\simeq246GeV$ which is the vev of the Higgs field \cite{ref:Halzen}.
Lagrangian (\ref{eq:ElectronLag}) then becomes
\begin{equation}
\mathcal{L}_Y^e(x)=-\frac{g_e}{\sqrt{2}}v(\bar{e}_Le_R+\bar{e}_Re_L)-\frac{g_e}{\sqrt{2}}(\bar{e}_Le_R+\bar{e}_Re_L)H(x)
\end{equation}
Here we can make the identification $m_e=g_ev/\sqrt{2}$ which gives as the electron mass.

For neutrinos in principle the same mechanism is valid, so that we start from the Lagrangian
\begin{equation}
\label{eq:nuLag}
\mathcal{L}_Y^{\nu_e}(x)=-g_{\nu_e}(\bar{\psi}_{eL}\phi_C \nu_{eR}+\bar{\nu}_{eR}\phi_C \psi_{eL})
\end{equation}
where
\begin{equation}
\phi_C(x)=i\tau_2\phi*(x)=\begin{pmatrix}\bar{\phi}^0(x) \\ -\phi^-(x)\end{pmatrix}
\end{equation}
is the charge conjugated field. Through SSB we obtain $\phi_C(x)=\begin{pmatrix}(v+H(x))/\sqrt{2}\\0\end{pmatrix}$, such that Lagrangian (\ref{eq:nuLag}) can be rewritten as
\begin{equation}
\label{eq:nuMassLag}
\mathcal{L}_Y^{\nu_e}(x)=-\frac{g_{\nu_e}}{\sqrt{2}}v(\bar{\nu}_{eL}\nu_{eR}+\bar{\nu}_{eR}\nu_{eL})-\frac{g_{\nu_e}}{\sqrt{2}}(\bar{\nu}_{eL}\nu_{eR}+\bar{\nu}_{eR}\nu_{eL})H(x)
\end{equation}
so that finally we obtained a (Dirac) mass for the neutrino, setting $g_{\nu_e}v/\sqrt{2}=m_{\nu_e}$
\begin{equation}
\mathcal{L}_{mass}^{\nu_e}=-m_{\nu_e}\bar{\nu}_{eL}\nu_{eR}+h.c.
\end{equation}

It is easy to check that the Standard Model Lagrangian with Yukawa interaction (\ref{eq:nuMassLag}) is invariant under the global transformation
\[
\nu_{eL}(x)\rightarrow e^{i\theta}\nu_{eL}(x),\quad\nu_{eR}(x)\rightarrow e^{i\theta}\nu_{eR}(x),
\quad e_L(x)\rightarrow e^{i\theta}e_L(x),\quad e_R(x)\rightarrow e^{i\theta}e_R(x)
\]
where $\theta$ is an arbitrary constant. Thus in the SM with massive neutrinos the total lepton number L is conserved  \emph{and $\nu_i(x)$ is the Dirac field of neutrinos (L=1) and antineutrinos (L=-1)}.

Although this mechanism of generation of neutrino masses is legitimate, if we look at the value of the Yukawa coupling $g_{\nu_e}$ it is many order of magnitudes smaller than the couplings for the other fermions \cite{ref:Bil}:
\[
2\cdot10^{-13}\leq g_{\nu_e}\leq10^{-12} \quad \text{while}\quad g_t\simeq7\cdot10^{-1}, g_b\simeq2\cdot10^{-2}, g_{\tau}\simeq7\cdot10^{-3}
\]
It looks extremely implausible that they have a common origin. This is also true if we think about smallness of neutrino masses: even if their absolute values are still unknown from neutrino oscillation and cosmological data the mass of the heaviest neutrino has been estimated to be
\[
5\cdot10^{-2}eV\simeq\biggl(\sqrt{\Delta m^2_A}\biggr)\leq m_3\leq\biggl(\frac{1}{3}\sum_im_i\biggr)\simeq3\cdot10^{-1}eV
\]
where $\Delta m^2_A$ is the atmosphere neutrino mass-squared difference \cite{ref:Bil}.

\subsection{Majorana neutrinos}
As we saw, Dirac mechanism for neutrino masses is not a completely exhaustive model.
A new beyond-the-Standard Model theory has been developed trying to account for neutrino masses generation in a more natural way. To be precise there is not a single theory but many \cite{ref:Bronc}\cite{ref:Bil}\cite{ref:front}. 
One of the most popular is the Weinberg effective Lagrangian: it allows the introduction of neutrino masses using only left-handed (Majorana) neutrino fields.
The advantage of this model is that it minimizes the number of degrees of freedom required. 
Dirac massive particles have four degrees of freedom ($\nu_{eL}$, $\nu_{eR}$ and their antiparticles), while for Majorana fermions they would be only two: $\nu_{eL}$ and $\nu_{eR}$. 

In order to build a Lorentz invariant mass term, a product of left-handed and right-handed components is required; this is achieved by the fact that the charge conjugated field
\begin{equation}
\nu_{eL}^c=C\bar{\nu}_{eL}^T
\end{equation}
is right-handed. A generic Majorana mass term may be written as
\begin{equation}
-\frac{1}{2}m_M\bar{\nu}_{eL}\nu_{eL}^c+h.c.
\end{equation}
It is straightforward to see that the above mass term is not invariant under a global $U(1)$ transformation
\begin{equation}
\nu_{eL}\rightarrow e^{-i\theta}\nu_{eL}
\end{equation}
corresponding to the non conservation of lepton number L.
Considering all neutrino flavors the Majorana mass term has the form
\begin{equation}
\label{eq:MajLag}
\mathcal{L}^M=-\frac{1}{2}\sum_{l',l}\bar{\nu}_{l'L}M_{l',l}^M\nu_{lL}^c+h.c.
\end{equation}
where $M^M$ is a complex $3\times3$ matrix. It can be diagonalized by the transformation $M^M=UmU^T$, with $UU^{\dagger}=1$ and  $m_{ik}=m_i\delta_{ik},\quad m_i>0$.
Lagrangian (\ref{eq:MajLag}) thus becomes
\begin{equation}
\mathcal{L}^M=-\frac{1}{2}\sum_{i=1}^3m_i\bar{\nu}_i\nu_i
\end{equation}
where $\nu_i$ are mass eigenstates with eigenvalues $m_i$.
\begin{equation}
\nu_i=\sum_lU_{il}^{\dagger}\nu_{lL}+\sum_l\bigl(U_{il}^{\dagger}\nu_{lL}\bigr)^c
\end{equation}
It's straightforward to verify that field $\nu_i$ satisfies Majorana condition $\nu_i=\nu_i^c$, and that the flavor neutrino fields $\nu_{lL}$ are given by a superposition of mass eigenstates $\nu_i$, whose contributions are weighted by coefficients $U_{il}$. 
\begin{equation}
\label{eq:nuMix}
\nu_{lL}=\sum_{i=1}^3U_{li}\nu_{iL}
\end{equation}
The matrix \emph{U} is called PMNS matrix.

\subsubsection{Weinberg effective Lagrangian}
In order to generate a Majorana mass term like (\ref{eq:MajLag}) Weinberg proposed a beyond the SM mechanism based on the effective Lagrangian approach.
Since we are dealing only with left-handed neutrino field $\nu_{lL}$ instead of the lepton doublet $\psi_lL$, it is not possible to generate Majorana masses through the usual coupling with the Higgs field and the subsequent symmetry breaking. 
One way to proceed is through the $SU_L(2)\times U_Y(1)$ invariant
\begin{equation}
\label{eq:LorInv}
(\phi_C^{\dagger}\psi_{lL})\qquad(l=e, \mu, \tau)
\end{equation}
which after spontaneous symmetry breaking assumes the form
\begin{equation}
(\phi_C^{\dagger}\psi_{lL})_{SSB}=\frac{v}{\sqrt{2}}\nu'_{lL}
\end{equation}
From (\ref{eq:LorInv}) the only possible effective Lagrangian which generates neutrino masses is
\begin{equation}
\label{eq:EffLag}
\mathcal{L}_{eff}=\frac{1}{\Lambda}\sum_{l',l}(\phi_C^{\dagger}\psi_{l'L})X_{l'l}(\phi_C^{\dagger}\psi_{lL})^c+h.c.
\end{equation}
where $X'$ is a symmetrical, non diagonal matrix.
The factor $1/\Lambda$ has been introduce since the mass dimension of (\ref{eq:EffLag}) is five, while a Lagrangian should have mass dimension four, and indeed $\Lambda$ has dimension \emph{M}.
The parameter $\Lambda$ is important since it can be interpreted as the mass scale at which new physics appears.

After SSB Lagrangian (\ref{eq:EffLag}) the following neutrino mass term emerges
\begin{equation}
\label{eq:EffMass}
\mathcal{L}^M=-\frac{v^2}{2\Lambda}\sum_{l',l}\bar{\nu}_{l'L}X_{l'l}\nu_{lL}^c+h.c.
\end{equation}
in which we can identify the Majorana matrix
\begin{equation}
M_M=\frac{v^2}{\Lambda}X. 
\end{equation}
The matrix \emph{X} can be diagonalized by the unitary matrix \emph{U}: $X=UxU^T$.
In this case Lagrangian (\ref{eq:EffMass}) becomes
\begin{equation}
\mathcal{L}^M=-\frac{1}{2}\sum_{i=1}^3m_{iL}\bar{\nu}_i\nu_i
\end{equation}
where
\begin{equation}
\label{eq:nuC}
\nu_i(x)=\nu_i^c(x)=C\bar{\nu}_i^T(x)
\end{equation}
is the field of  Majorana neutrino with mass 
\begin{equation}
\label{eq:neutMass}
m_{i}=\frac{v}{\Lambda}(vx_i)
\end{equation}.

From (\ref{eq:neutMass}) we see that neutrino masses are characterized by the two parameters \emph{v} and $\Lambda$: this can  provides an explanation for the smallness of neutrino masses with respect to other fermions. The suppression is by a factor
\begin{equation}
\frac{v}{\Lambda}=\frac{\text{EW scale}}{\text{scale of a new physics}}
\end{equation}
which is naturally much smaller than one.
If we assume the mass of the heaviest neutrino (mass) field to be $m_3\simeq5\times10^{-2}eV$ and $x_3\leq1$ (although parameters $x_i$ are unknown) we obtain the estimate $\Lambda\leq10^{15}GeV~\cite{ref:Bil}.$

\subsubsection{Type-I Seesaw mechanism}
The Standard Model for three generations of neutrinos can be extended to contain \emph{m} right-handed neutrinos $\nu_R$ building a gauge invariant renormalizable Lagrangian $\mathcal{L}=\mathcal{L}_{SM}+\mathcal{L}_{\nu_R}$ 
\begin{equation}
\label{eq:renormLag}
\mathcal{L}=i\bar{\nu}_L\sigma^{\mu}\partial_{\mu}\nu_L+i\bar{\nu}_R\sigma^{\mu}\partial_{\mu}\nu_R-\bar{\nu}_LM_D\nu_R-\frac{1}{2}\bar{\nu}^c_LM_L\nu_L-\frac{1}{2}\bar{\nu}_R^cM_R\nu_R+h.c.
\end{equation}
It contains a $m\times3$ complex Dirac mass matrix $M_D$ together with Majorana mass matrices $M_L$ and $M_R$ for left-handed and right-handed neutrinos, which are respectively $m\times m$ and $3\times3$ symmetric matrices \cite{ref:Fuji}. 
Notice that Majorana mass terms are not produced through Higgs mechanism and SSB.
Writing the complete mass term for $3+m$ neutrino generations results in 
\begin{equation}
\mathcal{L}_{mass}=-\frac{1}{2}\begin{pmatrix}\bar{\nu}_L \bar{\nu}_R^c\end{pmatrix}\begin{pmatrix}M_L & M_D\\ M_D^T & M_N\end{pmatrix}\begin{pmatrix}\nu_L^c\\ \nu_R\end{pmatrix}+h.c.
\end{equation}
where $\nu=(\nu_L, \nu_R^c)^T$ is a $(3+m)$-dimensional vector. In case of type-I seesaw mechanism the choice is $M_L=0$. The mass matrix can be diagonalized by unitary matrix $U$ of dimension $3+m$ to give 
\begin{equation}
U^T\begin{pmatrix}0 & M_D\\ M_D^T & M_R\end{pmatrix}U=\begin{pmatrix}M_l & 0 \\ 0 & M_h\end{pmatrix}
\end{equation}
$M_l$ and $M_h$ are $3\times3$ and $m\times m$ real diagonal matrices. If the mass eigenvalues of $M_R$ are much higher than the scale of electroweak symmetry breaking \emph{v}, we obtain three light neutrinos $\nu_l$ and \emph{m} heavy neutrinos, \emph{N}.
\begin{equation}
\mathcal{L}_{mass}=-\frac{1}{2}\bar{\nu}_lM^l\nu_l-\frac{1}{2}\bar{N}M^hN
\end{equation}
In this way values for LH neutrino masses compatible with experimental bounds are obtained. Notice that this does not imply that $M_D$ has to be small compared to the electroweak scale, and thus the unnaturally small coupling $g_{\nu}$ is avoided.

Weinberg effective Lagrangian arises naturally from Lagrangian (\ref{eq:renormLag}) by integrating out the heavy Majorana neutrino fields $\nu_{iR}$ and performing a power series expansion in $1/M_R$ of the form
\begin{equation}
\mathcal{L}_{eff}=\mathcal{L}_{SM}+\frac{1}{M_R}\mathcal{L}^{d=5}+\frac{1}{M_R^2}\mathcal{L}^{d=6}+\dots
\equiv \mathcal{L}_{SM}+\delta\mathcal{L}^{d=5}+\delta\mathcal{L}^{d=6}+\dots
\end{equation}
Here the $\delta\mathcal{L}^{d=5}$ effective Lagrangian corresponds to the Weinberg effective Lagrangian (\ref{eq:EffLag}) so that we can find the relation $\Lambda\equiv M_R$.
This means that the scale of new physics $\Lambda$ is defined by the masses of heavy Majorana fields $N$.
The SM is an effective theory which properly describes the physics at energies below the mass scale $M_R$.

\subsubsection{Light sterile neutrinos}
It could be that some $n_s<m$ eigenvalues of the Majorana matrix $M_R$ are not higher than the electroweak scale; in this case there will be additional $n_s$ light states and we will have $3+n_s$ light neutrinos.
Indications in favor of sterile neutrinos were obtained in different short baseline experiments, such as LSND and MiniBooNE, in which some anomalies were observed which cannot be explained by the three-flavor oscillation model.
The existing data could be explained through the introduction of a fourth massive (sterile) neutrino. For a more detailed discussion see~\cite[pp.~39-42]{ref:pdg}.  

\subsection{Other Seesaw models}
Apart from Type-I seesaw mechanism there is a variety of models able to generate Majorana mass terms and to account for the smallness of neutrino masses.
The fundamental ones are called Type-II and Type-III seesaw models, which are then extended and hybridized.
In Type-II models the Weinberg operator is generated by an additional Higgs triplet field $\Delta$ (different from the SM Higgs \emph{H}), while Type-III models introduce a fermion SU(2) triplet $\Sigma_L$. For a more detailed discussion see~\cite{ref:front}.

In addition to these theories there is a large number of effective terms that can each generate their neutrino mass. Weinberg term is the only possible 5-dimensional term using the Gauge group of the SM. However there are higher-dimensional terms e.g. 7-dimensional terms which suppress the neutrino mass by $\sim1/M^3$.
One of the advantage of these models is that the mass scale M is lower than in three-level models. On the other side in this way the number of possible neutrino mass models quickly become very large.

\section{Nutrinoless double beta decay}
Double beta decay ($2\nu\beta\beta$) is the simultaneous beta decay of two neutrons in a nucleus: $(A, Z)\rightarrow(A, Z+2)+2e^-+2\bar{\nu}_e$.
It is a second order process that has been observed in even-even nuclei in which single beta decay is energetically forbidden. As we saw, some models that extend the SM predict the existence of a \emph{neutrinoless} double beta decay ($0\nu\beta\beta$)
\begin{equation}
(A, Z)\rightarrow(A, Z+2)+2e^-
\end{equation}
This process could be observed whether other possible decay channels are forbidden, and several ongoing experiments nowadays are trying to search for it. This will prove the existence of Majorana neutrinos and will also give important information about neutrino mass spectrum.
\begin{wrapfigure}{l}
{0.35\textwidth}
\includegraphics[width=0.3\textwidth]{Neutrinoless.png}
\caption{$0\nu\beta\beta$-decay.}
\vspace{-10pt}
\end{wrapfigure}
This process has a unique signature, since the emitted electrons carry all the energy of the decay. The energy spectrum of the two emitted electrons therefore is a sharp peak centered around the Q-value of the decay, as shown in Fig.~\ref{fig:Nspectrum}.
\begin{figure}
\centering
\includegraphics[width=0.5\textwidth]{Nspectrum.png}
\caption{Energy spectra for double $\beta$-decay and neutrinoless double $\beta$-decay.}
\label{fig:Nspectrum}
\end{figure}

\subsection{Theory}
The theory on $0\nu\beta\beta$-decay is based on the assumption that the CC interaction is the SM interaction
\begin{equation}
\mathcal{L}^{CC}=-\frac{g}{2\sqrt{2}}j_{\alpha}^{CC}W^{\alpha}+h.c.
\end{equation}
with
\begin{equation}
j_{\alpha}^{CC}=2\bar{\nu}_{lL}\gamma_{\alpha}l_L+j_{\alpha}^{CCquark}
\end{equation}
where $j_{\alpha}^{CCquark}$ is the quark charged current, $W^{\alpha}$ is the charged vector bosons field and $g$ is the constatnt of electroweak interaction. The flavor neutrino field $\nu_{lL}$ is given by the mixing relation (\ref{eq:nuMix}) of Majorana neutrinos (\ref{eq:nuC}).

The effective Hamiltonian of $\beta$-decay is 
\begin{equation}
\mathcal{H}=\frac{G_F}{\sqrt{2}}2\bar{e}_L\gamma_{\alpha}\nu_{eL}j^{\alpha}+h.c.
\end{equation}
In order to find the decay rate of this process, the matrix element for neutrinoless double $\beta$-decay has to be calculated in the second order of the perturbation theory in $G_F$. Going directly to the result, the total decay rate is
\begin{equation}
\Gamma^{0\nu}=\frac{ln(2)}{T^{0\nu}_{1/2}}=|m_{\beta\beta}|^2|M^{0\nu}|^2G^{0\nu}(Q,Z)
\end{equation}
where the matrix element is given by the product of the effective Majorana mass $m_{\beta\beta}$ and the nuclear matrix element $M^{0\nu}$, and $G^{0\nu}(Q,Z)$ is the phase space factor, which includes effects of Coulomb interaction between electrons and the daughter nucleus. For a complete calculation of the decay rate see~\cite{ref:Bil}.

A particular focus has to be set on the effective Majorana mass
\begin{equation}
\label{eq:BBmass}
m_{\beta\beta}=\sum_{i=1}^3U_{ek}^2m_k=|c_{12}^2c_{13}^2e^{2i\alpha_1}m_1+c_{13}^2s_{12}^2e^{2i\alpha_2}m_2+s_{13}^2m_3|
\end{equation}
since it depends on neutrino masses (still unknown) and on the matrix elements of the PMNS matrix, which are extracted from neutrino oscillation experiments.

Studying solar neutrinos we get $\Delta m_S^2\equiv\Delta m_{12}^2>0$, while from atmospheric neutrino experiments $\Delta m_A^2$ is derived.
At this point however there are two possibilities, since we don't know values of neutrino masses but only their squared differences:
\begin{enumerate}
\item Normal ordering (NO)
\begin{equation}
m_3>m_2>m_1
\end{equation}
\item Inverted ordering (IO)
\begin{equation}
m_2>m_1>m_3
\end{equation}
\end{enumerate}
In the first case (NO) $\Delta m_A^2\equiv\Delta m_{23}^2$ and
\begin{equation}
m_2=\sqrt{m_1^2+\Delta m_S^2} \qquad m_3=\sqrt{m_1^2+\Delta m_S^2+\Delta m_A^2}
\end{equation}
while for inverted ordering $\Delta m_A^2\equiv\Delta m_{13}^2$ and
\begin{equation}
m_1=\sqrt{m_3^2+\Delta m_A^2} \qquad m_2=\sqrt{m_3^2+\Delta m_S^2+\Delta m_A^2}
\end{equation}

For what concerns the PMNS matrix, if we are dealing with Dirac neutrinos it is characterized by three mixing angles $\theta_{12}, \theta_{23}, \theta_{13}$ and one phase $\delta$. In the standard parametrization it has the form
\begin{equation}
U^D=\begin{pmatrix}c_{13}c_{12}&c_{13}s_{12}&s_{13}e^{-i\delta}\\-c_{23}s_{12}-s_{23}c_{12}s_{13}e^{i\delta}&c_{23}c_{12}-s_{23}s_{12}s_{13}e^{i\delta}&c_{13}s_{23}\\s_{23}s_{12}-c_{23}c_{12}s_{13}e^{i\delta}&-s_{23}c_{12}-c_{23}s_{12}s_{13}e^{i\delta}&c_{13}c_{23}\end{pmatrix}
\end{equation}
where $c_{ij}\equiv cos(\theta_{ij})$ and $s_{ij}\equiv sin(\theta_{ij})$.

If instead neutrinos are Majorana particles, two additional phases $\alpha_1$ and $\alpha_2$ enter in the mixing matrix
\begin{equation}
U^M=U^DS^M(\alpha)
\end{equation}
where
\begin{equation}
s^M(\alpha)=\begin{pmatrix}e^{i\alpha_1}&0&0\\0&e^{i\alpha_2}&0\\0&0&1\end{pmatrix}
\end{equation}
These two complex phases $\alpha_{1,2}$ however cannot be measured.

In the table below values of the oscillation parameters in case of NO (IO) can be found. They are obtained from Super-Kamiokande data analysis~\cite[p.~38]{ref:pdg}. 
\[
\begin{array}{lcc}
\toprule
&\text{Normal Ordering}&\text{Inverted Ordering}\\
\midrule
\sin^2\theta_{12}/10^{-1}&3.10_{-0.12}^{+0.13}&3.10_{-0.12}^{+0.13}\\
\theta_{12}/\si{\degree}&33.82_{-0.76}^{+0.78}&33.82_{-0.76}^{+0.78}\\
\sin^2\theta_{23}/10^{-1}&5.58_{-0.33}^{+0.20}&5.63_{-0.26}^{+0.19}\\
\theta_{23}/\si{\degree}&48.3_{-1.9}^{+1.2}&48.6_{-1.5}^{+1.1}\\
\sin^2\theta_{13}/10^{-2}&2.241_{-0.065}^{+0.066}&2.261_{-0.064}^{+0.067}\\
\theta_{13}/\si{\degree}&8.61_{-0.13}^{+0.13}&8.65_{-0.12}^{+0.13}\\
\delta_{CP}/\si{\degree}&222_{-28}^{+38}&285_{-26}^{+24}\\
\Delta m^2_{21}/10^{-5}eV^2&7.39_{-0.20}^{+0.21}&7.39_{-0.20}^{+0.21}\\
\Delta m_{32}^2/10^{-3}eV^2&2.449_{-0.030}^{+0.032}&-2.509_{-0.032}^{+0.032}\\
\bottomrule
\end{array}
\]
The effective Majorana mass is strongly dependent on the neutrino mass spectrum. Starting from Eq. (\ref{eq:BBmass}) and experimental values of oscillation parameters we can derive upper and lower bounds for $m_{\beta\beta}$~\cite{ref:ver}.
\begin{itemize}
\item For Normal Ordering the following approximation is valid
	\begin{equation}
	m_1\ll\sqrt{\Delta m_S^2} \quad m_2\simeq\sqrt{\Delta m_S^2} \quad m_3\simeq\sqrt{\Delta m_A^2}
	\end{equation}
	resulting in 
	\begin{equation}
	\SI{1.4}{meV} \leq m_{\beta\beta} \leq \SI{3.6}{meV}
	\end{equation}
\item In the case of Inverted Ordering instead
	\begin{equation}
	m_1\simeq\sqrt{\Delta m_A^2} \quad m_2\simeq\sqrt{\Delta m_A^2} \ll\sqrt{\Delta m_A^2}
	\end{equation}
	resulting in
	\begin{equation}
	\SI{20}{meV}\leq m_{\beta\beta} \leq \SI{49}{meV}
	\end{equation}
\end{itemize}
\begin{figure}[h]
\centering
\includegraphics[width=0.9\textwidth]{BBboundaries.png}
\caption{Predictions on $m_{\beta\beta}$ from neutrino oscillation versus the lightest neutrino mass $m_0$ in case of Normal (blue region) and Inverted (red region) Ordering. The excluded region by cosmological data is presented in yellow. The $3\sigma$ values of neutrino oscillation parameters are considered~\cite{ref:ver}}
\label{fig:BBbound}
\end{figure}
In Figure~\ref{fig:BBbound} prediction on the effective Majorana mass is plotted as a function of the lightest neutrino mass $m_0$. The two Majorana phases $\alpha_{1,2}$ are assumed to be arbitrary. From cosmological observations~\cite{ref:ver} it is derived the constraint 
\begin{equation}
\sum_{k=1}^3m_k\leq m_{astro}=\SI{110}{meV}
\end{equation}

It's clear that the observation of $0\nu\beta\beta$-decay would confirm the Majorana nature of neutrinos and also provide important  information on the neutrino mass spectrum.

In the following section highlights on some of the present and past experiments on neutrinoless double beta decay are presented.

\subsection{Experiments}
The search for $0\nu\beta\beta$-decay is carried out through several large scale experiments.General requirements for the employed detectors are high purity, low background and huge dimensions of the samples, because of the small cross-section of the process.
Experiments we are going to quickly discuss use three different target materials: Xenon-136 (EXO and KamLAND-Zen), Germanium-76 (GERDA), Tellurium-130 (CUORE) and Selenium-82 (Cupid-0).

\subsubsection{EXO}
The Enriched Xenon Observatory is an experiment in operation at the Waste Isolation Pilot Plant (WIPP) at a depth of about \SI{1624}{m} water equivalent near Carlsbad in New Mexico (USA).
The current experiment EXO-200 consists of \SI{200}{Kg} of liquid Xenon (LXe) enriched to 80.6\% in \ce{^{136}Xe} isotope.  It was the first experiment to observe double beta decay in Xenon ($T_{1/2}\sim 2\cdot10^{21}\SI{}{yr}$).
The LXe fills a TPC vessel, so that when a charged particle traverses it, a signal is produced through ionization. The ions drift towards the wire grids, which provide a 2D location. Scintillation light is also produced. The third coordinate is reconstructed from the time difference between light signal and ionization signal. The amount of light is proportional to the event's energy.

Xenon-36 has a Q-value of \SI{2458}{keV} and therefore gamma rays from naturally-occurring radioactive isotopes are a background that can make the $0\nu\beta\beta$-decay difficult or impossible to detect.
However, we are able to achieve great energy resolution through collecting both ionization electrons and scintillation light from Xenon and furthermore tagging of the daughter Barium ion in principle could allow for complete background rejection.

In December 2018 EXO-200 completed data taking with the upgraded detector ("Phase II", May 2016 to December 2018) after collecting a total exposure of \SI{234.1}{Kg\cdot yr}, similar to that of its first run ("Phase I", September 2011 to February 2014).
The lower limit on the \ce{^{136}Xe} $0\nu\beta\beta$ half-life is $T_{1/2}>1.7\cdot10^{25}\SI{}{yr}$ ($T_{1/2}>4.3\cdot10^{25}\SI{}{yr}$) at 90\% CL in Phase I (Phase II), whereas the combined limit is $T_{1/2}>3.5\cdot10^{25}\SI{}{yr}$. From these results the upper limit on the Majorana mass is $\langle m_{\beta\beta}\rangle<(93-286)\SI{}{meV}$ \cite{ref:exo}.

\begin{figure}[h]
\centering
\includegraphics[width=0.7\textwidth]{exo200.png}
\caption{90\% CL exclusion sensitivity reach to $\langle m_{\beta\beta}\rangle$ as a function of the lightest neutrino mass for NO and IO. The dashed contours of the NO and IO result from the unknown Majorana phases \cite{ref:nexo}.}
\end{figure}

A larger experiment called nEXO (next EXO) is the planned tonne-scale successor to EXO-200, it will be located at SNOLAB in Canada.
The experiment will reach a half-life sensitivity of $1.35\cdot10^{28}\SI{}{yr}$ at 90\% CL in 10 years of data taking \cite{ref:nexo}.

\subsubsection{KamLAND-Zen}
KamLAND (Kamioka Liquid scintillator Anti-Neutrino Detector) is an experiment searching for the $0\nu\beta\beta$-decay of \ce{^{136}Xe} using the KamLAND facility. It is located \SI{1000}{m} underground in the Kamioka mine in Gifu prefecture in Japan.
\begin{figure}
\centering
\includegraphics[width=0.55\textwidth]{KamLAND.png}
\caption{KamLAND-Zen detector. In the centre of the detector there is the inner balloon (IB) filled with enriched \ce{^{136}Xe}-loaded LS \cite{ref:kam}.}
\label{fig:kamDet}
\end{figure}
The KamLAND detector is a 18m-diameter spherical thank containing a 13m in diameter balloon with \SI{1000}{ton} of ultra-pure liquid scintillator (LS). Outside the tank there is a \SI{3200}{ton} water Cherenkov detector for muon identification.
KamLAND-Zen is a minor modification of KamLAND, accomplished just by putting in the center of the detector a small balloon (inner balloon, IB) of \SI{3.08}{m} in diameter, filled with \SI{320}{Kg} of enriched Xe (with 91\% \ce{^{136}Xe}) dissolved in liquid scintillator.

KamLAND-Zen started data acquisition in October 2011 until June 2012 (phase I) and after an exposure of \SI{89.5}{Kg\cdot yr} the 90\% CL limit on half-life resulted in $T_{1/2}>1.9\cdot10^{25}\SI{}{yr}$.
The phase II started in November 2013 until October 2015 and obtained the limit  $T_{1/2}>9.2\cdot10^{25}\SI{}{yr}$ which, combined with the one in phase I lead to  $T_{1/2}>1.07\cdot10^{26}\SI{}{yr}$ at 90\% CL.

The upper limit on Majorana mass resulted in $\langle m_{\beta\beta}\rangle<(61-165)\SI{}{meV}$, as shown in  Figure~\ref{fig:kamSpec}, while the limit on the lightest neutrino mass is obtained as $(180-480)\SI{}{meV}$ at 90\% CL \cite{ref:kam}.
\begin{figure}
\centering
\includegraphics[width=0.6\textwidth]{KamSpectrum.png}
\caption{Effective Majorana mass as a function of the lightest neutrino mass. The dark shaded regions are from the best-fit neutrino oscillation parameters for NO and IO mass spectrum. Light shaded regions are the $3\sigma$ ranges of the oscillation parameter uncertainties. Horizontal bands indicate 90\% CL upper limits on the $\langle m_{\beta\beta}\rangle$ from KamLAND-Zen and other experiments \cite{ref:kam}.} %%%not up to date
\label{fig:kamSpec}
\end{figure}

A second period of data acquisition started in January 2019 with an enlarged IB which now contains \SI{750}{Kg} of Xenon. The new detector is called KamLAND-Zen 800.
The backgrounds from Xenon loaded liquid scintillator vessel has been reduced by a factor 10. The very preliminary analysis of 132.7 days data gives lower limit for the $0\nu\beta\beta$ of $4\cdot 10^{25}\SI{}{yr}$ (90\% CL), and the sensitivity is $8\cdot 10^{25}$ years \cite{ref:kam800}. 

The further improvement of the detector will lead to KamLAND2-Zen, which will use \SI{1}{ton} of Xenon and is scheduled to start in 2027.

\subsubsection{GERDA}
The GERmanium Detector Array experiment worked with high-purity germanium detectors made out of material enriched in \ce{^{76}Ge} to $\sim87\%$. It is located at the Laboratori Nazionali del Gran Sasso (LNGS) of INFN, Italy,  under \SI{3500}{m} of water equivalent of rock.
One of the advantage in using germanium detectors is their energy resolution, which guarantees a very clear signature of $0\nu\beta\beta$-decay signal. is Q-value is \SI{2039.06}{keV}
\begin{figure}
\centering
\includegraphics[width=0.95\textwidth]{GERDA.png}
\caption{Experimental setup for the GERDA detector \cite{ref:gerLay}.}
\label{fig:GerLayout}
\end{figure}
In Figure~\ref{fig:GerLayout} is displayed the set up of the detector. The core of the experiment is an array of broad energy germanium (BEGe) detectors suspended in strings into a cryostat filled with liquid Argon (LAr), which provides both shielding and cooling.
The cryostat is surrounded by a water tank equipped with photomultimpliers to detect the residual cosmic muons (not stopped by the rock). The muon veto system is complemented by scintillators panels on the top of the clean room.
A cylindrical volume around the array is instrumented with photosensors, in order to detect scintillation light in the LAr.

In this experiment detector and the source coincide, maximizing the detection efficiency.
The two electrons emitted in a double $\beta$-decay have a range in germanium of the order of \SI{1}{mm}: they deposit their energy in a small volume of the detector and thus produce highly localized events. In contrast, $\gamma$ rays of similar energy mostly interact via Compton scattering and can produce events with several separated energy depositions. Events in which more than one germanium detectors is fired are therefore identified as background.

Phase I of GERDA collected \SI{23.5}{Kg\cdot yr} of exposure between November 2011 and September 2013. After an overall upgrade GERDA Phase II started data acquisition in December 2015 until November 2019, with a total exposure of \SI{103.7}{Kg\cdot yr}.
The analysis of Phase II data yields no indication of $0\nu\beta\beta$-decay and set a lower limit of $T_{1/2}>1.5\cdot10^{26}\SI{}{yr}$ at 90\% CL. Combining all the results from Phase I, II , with a total exposure of \SI{127.2}{Kg\cdot yr} led to $T_{1/2}>1.8\cdot10^{26}\SI{}{yr}$ at 90\% CL.
\begin{figure}
\centering
\includegraphics[width=0.6\textwidth]{Tgerda.png}
\caption{Circles: lower limit (90\% CL) on the $0\nu\beta\beta$ decay half-life of \ce{^{76}Ge} set by GERDA as a function of the exposure. Triangles: median expectation in the assumption of no signal. \cite{ref:gerda}}
\label{fig:Tgerda}
\end{figure}
The $T_{1/2}$ limit can be converted into an upper limit on the effective Majorana mass $m_{\beta\beta}<(79-180)\SI{}{meV}$ at 90\% CL \cite{ref:gerda}.

A new experiment, called LEGEND-200 (Large Enriched Germanium Experiment for Neutrinoless double beta Decay) has taken over the GERDA infrastracture at LNGS and should start data taking in 2021, reaching in 5 years a sensitivity of $10^{27}$ and beyond.

\subsubsection{CUPID-0}
The project CUPID (CUORE Upgrade with Particle Identification) was conceived for the measurement of neutrinoless beta decay with scintillating bolometers. It is located at Laboratori Nazionali del Gran Sasso.
It is based on the infrastructure of the CUORE experiment (also at LNGS), which was a ton-scale bolometric detector designed to search for $0\nu\beta\beta$-decay in \ce{^{130}Te}.

A blometer consists of a crystal operating as a cryogenic calorimeter, around \SI{0.01}{K}. At this temperature a minimum energy release in it produces a temperature variation. By measuring this thermal variation it is possible to reconstruct the energy released.

CUORE used a detector made of 988 \ce{TeO2} crystals, which have important characteristic of emitting very faint light signals, together with the production of heat signals. The emitted light is measured by dedicated (bolometric) detectors, able to work at cryogenic temperatures. In this way the natural background radiation (e.g. $\alpha$ particles) can be reduced.
The choice of Tellurium was made since it has a Q-value for $0\nu\beta\beta$-decay of \SI{2527.518}{keV}, above most of the natural radioactive background \cite{ref:cuore}.

CUPID will increase the active mass, through isotopic enrichment and also decrease the background by two orders of magnitude with respect to CUORE. The experiment used \ce{^{82}Se}, which has one of the highest Q-values for $0\nu\beta\beta$-decay (\SI{2997.9}{keV}). This choice allows to work in a region barely affected by the $\beta/\gamma$ background from natural radioactivity.
Selenium was used to grow 24 cylindrical \ce{Zn^{82}Se} crystals 95\% enriched in \ce{^{82}Se}, arranged in five towers (see Figure~\ref{fig:cupid}).
\begin{figure}
\centering
\includegraphics[width=0.8\textwidth]{cupid.png}
\caption{a) Lateral view of a single module: the \ce{ZnSe} is placed between two light detectors (Ge-LD), not visible in this scheme because of their small thickness (about \SI{170}{nm}); all the detectors are held in the copper structure. Top (b) and 3D (c) views of the CUPID-0 detector. d) 3D view of the CUPID-0 detector hosted in the CUORE-0 cryostat. Source: https://www.lngs.infn.it/en/cupid}
\label{fig:cupid}
\end{figure}
The CUPID-0 detector has been taking data from June 2017 to December 2018, collecting a \ce{^{82}Se} exposure of \SI{5.29}{Kg\cdot yr}. The extrapolated limit on the half-life for $0\nu\beta\beta$-decay in Selenium was $T_{1/2}>3.5\cdot10^{24}\SI{}{yr}$ corresponding to $m_{\beta\beta}<(311-638)\SI{}{meV}$ \cite{ref:cupid}. 

CUORE experiment after a total exposure of \SI{372.5}{Kg\cdot yr} obtained a 90\% CL lower limit of $3.2\cdot10^{25}\SI{}{yr}$, which leads to an upper limit for the effective Majorana mass of $(75-350)\SI{}{meV}$ \cite{ref:cuore}.

\section{Collider experiments}
We just discussed about several experiments which are trying to observe $0\nu\beta\beta$-decay. Complementary to them are on-going searches for lepton number violating processes at collider experiments such as LEP, Belle, LHCb and ATLAS/CMS.

For Majorana neutrinos below the $M_W$ mass scale, lepton number violating processes may manifest in numerous way, including rare decays of mesons, baryons, $\mu$ and $\tau$ leptons and even SM electroweak bosons. An example of an L-violating process is displayed in Figure \ref{fig:Lviol}.
\begin{figure}
\centering
\includegraphics[width=0.7\textwidth]{Bdecay.png}
\caption{$B^-$ meson decay to L-violating final state via heavy Majorana N \cite{ref:front}.}
\label{fig:Lviol}
\end{figure}
Indirectly, the presence of heavy Majorana neutrinos can appear in precision EW measurements as deviations from lepton flavor unitarity and universality,  and is ideally suited for $e^+e^-$ colliders.

For Majorana neutrinos with $M_N>M_W$ the most studied production mechanism at pp colliders is the L-violating Drell-Yan CC process shown in Figure \ref{fig:DY}.
\begin{figure}
\centering
\includegraphics[width=0.5\textwidth]{DYprocess.png}
\caption{Born diagram for heavy neutrino production via Drell-Yan process (for Type-I seesaw)  \cite{ref:front}.}
\label{fig:DY}
\end{figure}
The Type-II seesaw is characterized by heavy SU(2) triplet scalars, which result in new singly- and doubly-charged Higgs bosons. they cab be copiously produced in pairs via SM electroweak gauge interactions if kinematically accessible at collider energies.
For more details see \cite{ref:front}. 

\section{Conclusions}


\appendix

\section{}
Let us go then, you and I\ldots

\acknowledgments
This work was produced, supported and perpetrated by M. Bellacosa under
the auspices of the Italian Physical Society.

\begin{thebibliography}{0}
\bibitem{ref:cott} \BY{Cottingham~W.N. \atque Greenwood~D.A.} 
\TITLE {An introduction to the Standard Model of Particle Physics. Second edition.} Cambridge University Press, 2007.
\bibitem{ref:pdgBos} \BY{Tanabashi~M. \emph{et al.}} (Particle Data Group) \IN{Phys. Rev. D}{98}{030001} (2018) and 2019 update.
\bibitem{ref:wrap} \BY{Chappell~A.} \TITLE{CP violation and neutrino mass ordering at the T2K experiment} PhD thesis, University of Warwick (2019), pp. 5-8.
\bibitem{ref:ait2} \BY{Aitchison~I.J.R. \atque Hey~A.J.G.} \TITLE{Gauge Theories in Particle Physics. A practical introduction. Vol.2}, in \TITLE{Graduate student series in Physics}, edited by \NAME{Brewer~D.A.} (Institute of Physics Publishing, Bristol and Philadelphia) 2004, pp. 368-372.
\bibitem{ref:Halzen} \BY{Halzen~F. \atque Martin~A.D.} \TITLE{Quarks and Leptons: An Introductory Course in Modern Particle Physics} 
(John Wiley \& Sons.) 1984, pp. 338-340.
\bibitem{ref:Bronc} \BY{Broncano~A. , Gavela~M.B. \atque Jenkins~E.} \TITLE{The Effective Lagrangian for the Seesaw Model of Neutrino Mass and Leptogenesis}, 2006  [hep-ph/0210271].
\bibitem{ref:Fuji} \BY{Fujikawa~K.} \IN{Eur. Phys. J. C}{80}{2020}{285}.
\bibitem{ref:Bil} \BY{Bilenky~S.M.} \TITLE{Neutrinos: Majorana or Dirac?}, 2020 [arXiv:2008.02110v1].
\bibitem{ref:pdg} \BY{Zyla~P.A. \emph{et al.}} (Particle Data Group), \IN{Prog. Theor. Exp. Phys.}{2020} {083C01} 2020.
\bibitem{ref:ver} \BY{Vergados~J.D. , Ejiri~H. \atque Šimkovic~F.} \TITLE{Neutrinoless double beta decay and neutrino mass}, 2016 [arXiv:1612.02924v1].
\bibitem{ref:front} \BY{Cai~Y. , Han~T. , Li~T. \atque Ruiz~R.} \IN{Front. Phys.}{6}{40} 2018.
\bibitem{ref:exo} \BY{Anton~G. \emph{et al.}} (EXO-200 Collaboration), \IN{Phys. Rev. Lett.}{123}{16182} 2019.
\bibitem{ref:nexo} \BY{Adhikari~G. \emph{et al.}} \TITLE{nEXO: Neutrinoless double beta decay search beyond $10^{28}$ year half-life sensitivity}, 2021 [arXiv:2106.16243v1].
\bibitem{ref:kam} \BY{Shirai~J.} (for the KamLAND-Zen Collaboration) \IN{J. Phys.:Conf. Ser.}{888}{012031} 2017.
\bibitem{ref:kam800} \BY{Gando~Y.} (on the behalf of the KamLAND-Zen Collaboration) \IN{J. Phys.: Conf. Ser.}{1468}{012142} 2020.
\bibitem{ref:gerda} \BY{Agostini~M. \emph{et al.}} (GERDA Collaboration) \IN{Phys. Rev. Lett.}{125}{252502} 2020.
\bibitem{ref:gerLay} \BY{Majorovits~B.} (for the GERDA collaboration) \TITLE{Search for $0\nu\beta\beta$-decay with GERDA Phase II} https://slideplayer.com/slide/12699477/
\bibitem{ref:cuore} \BY{Adams~D.Q. \emph{et al.}} (CUORE Collaboration) \TITLE{Improved Limit on Neutrinoless Double Beta Decay in \ce{^{130}Te} with CUORE}, 2019 [arXiv:1912.10966v1].
\bibitem{ref:cupid} \BY{Azzolini~O. \emph{et al.}} \TITLE{Final result of CUPID-0 phase-I in the search for the \ce{^{82}Se} Neutrinoless Double Beta Decay}, 2019 [arXiv:1906.05001v1].

\end{thebibliography}

\end{document}