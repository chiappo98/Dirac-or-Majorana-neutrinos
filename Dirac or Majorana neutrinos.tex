\documentclass{subnucbo}
\usepackage[utf8]{inputenc}
\usepackage[english]{babel}
\usepackage{graphicx}
\usepackage{amsmath,amssymb}
\usepackage{float}
\usepackage{siunitx}
\usepackage{booktabs}
\usepackage{mathrsfs}

\begin{document}

\title{Dirac or Majorana neutrinos: an open question}

\author{F. Chiapponi} 

\instlist{\inst  Dipartimento di Fisica Universit\`a di Bologna, Via Irnerio 46 - 40126 Bologna, Italy} 

\acyear{2020--2021}

\maketitle

\begin{abstract}
In this paper I will present the state of the research on neutrinos, together with several still open questions. In particular our focus will be set on whether neutrinos are Dirac or Majorana particles. In the end we will describes the experiments that attempt to observe a neutrinoless double beta decay in even-even nuclei%~\ref{}
, which will prove the Majorana nature of these particles.
\end{abstract}

\section{An elusive particle}
\subsection{History}
The neutrino was first theorized in 1930 by Wolfgang Pauli in order to account for conservation of energy and momentum in beta decay.
At the beginning this process was though to proceed as
\[
^A_Z X\rightarrow ^A_{Z+1}Y+e^-
\]
However looking at the energy spectrum of emitted electrons (\emph{Fig.}~\ref{fig:betaD}) it was clear that it was due to a three body process, but then a particle was missed. For a two body decay the electrons' energy spectrum should form a sharp peak around 
\begin{equation}
E=\biggl(\frac{m^2_A-m^2_B+m^2_e}{2m_A}\biggr)c^2
\label{eq:Emax}
\end{equation}
This energy corresponds only to the maximum electron energy of \emph{Figure}~\ref{fig:betaD}.

\begin{figure}[h]
\centering
\includegraphics[width=0.6\textwidth]{energy_spectrum.png}
\caption{The beta decay spectrum of tritium ($^3_1H\rightarrow ^3_2He$). (\emph{Source}: G.M. Lewis, \emph{Neutrinos} (London: Wykeham, 1970))}
\label{fig:betaD}
\end{figure}

The problem of the missing energy could be solved introducing a new particles, which had to be neutral to conserve electric charge and extremely light, from the fact that the observed electron energies range up to the value given in equation (\ref{eq:Emax}).
Fermi called this particle \emph{neutrino}, which was finally discovered in 1956 by Cowan and Reines at the Savannah River nuclear reactor in South Carolina, where they set up a large tank of water and watched for inverse beta decay reactions (see section \ref{sec:prod}).

Today we know the existence of three flavors of neutrinos, associated to the corresponding charged leptons %~\ref{cottingham} (maybe inside caption)
:
\begin{table}[H]
\begin{tabular}{cccc}
\toprule
Lepton & Charge & Mean life (s) & Mass (MeV)\\
\midrule
$e^-$ & $-e$ & $\infty$ & 0.511\\
$\nu_e$ & 0 &  & 0\\
$\mu^-$ & $-e$ & $2.197\cdot10^{-6}$ & 105.658\\
$\nu_\mu$ & 0 & & 0\\
$\tau^-$ & $-e$ & $(291.0\pm1.5)\cdot10^{-15}$ & 1777\\
$\nu_\tau$ & 0 &  & 0\\
\bottomrule
\end{tabular}
\caption{Leptons.}
\label{tab:classification}
\end{table}
 
According to the Standard Model (SM) neutrinos are \emph{massless} fermions. Leptons in the table above all have Total Lepton Number $L=+1$, while their antiparticles have $L=-1$, together with opposite electric charge.

An interesting feature of weak interactions is that it does not conserve parity. Indeed, it was demonstrated by Madame Wu in 1957 that only left-handed component of electrons and neutrinos participate in the interaction, and the same for other flavors. Conversely antiparticles enter in interactions with right-handed component.

\subsection{Neutrinos production and interaction}
\label{sec:prod}
Neutrinos are produced in leptonic (\ref{eq:Lep}) and semileptonic decays (\ref{eq:semiL1} - \ref{eq:semiL3}) of particles involving weak interactions, but also in weak decays of unstable atomic nuclei. 
\begin{eqnletter}
\label{eq:lep}
\mu^- &\rightarrow& e^-+\bar{\nu}_e+\nu_\mu \label{eq:Lep}\\
\pi^+ &\rightarrow& \mu^++\nu_\mu \label{eq:semiL1}\\
K^+ &\rightarrow& \pi^0+\mu^++\nu_\mu \label{eq:semiL2}\\
&\hookrightarrow& \pi^0+e^++\bar{\nu}_e \label{eq:semiL3}
\end{eqnletter}
On Earth the flux of neutrinos is dominated by Solar neutrinos, produced in fusion processes inside the Sun, and neutrinos of atmospheric origin, due to the interaction of cosmic rays with the atmosphere. However they are copiously produced also in nuclear power plants, which often are used as neutrinos source for particle physics experiments.

Weak processes can be classified in charged current (CC) and neutral current (NC) interactions, whether or not there is a change in the electric charge between initial and final fermions. CC interactions are mediated by the charged $W^-$ and $W^+$ bosons, while NC interaction by the $Z$ boson, which is neutral.
These mediators are massive %\cite{pdg}:
\begin{gather}
M_W=80.379\pm0.012 GeV \\
M_Z=91.1876\pm0.0021 GeV
\end{gather}
It is this fact that makes interactions mediated by these bosons weak and short-ranged.%insert weak coupling and cross sections?

Being neutral leptons neutrinos interact only through weak interactions, and hence are very difficult to detect. 
On the other side they are very useful to study weak interactions since weak transition amplitude of processes in which they are involved is not overcame by strong or electromagnetic amplitudes.
Neutrinos can be detected through the inverse beta decay interaction
\begin{equation}
\bar{\nu}_e+p \rightarrow e^++n
\end {equation}
that is, observing the emitted charged lepton, usually by its Cherenkov emission. %verify
Another interaction channel is elastic scattering 
\begin{equation}
\nu_i+e^- \rightarrow \nu_i+e^- \quad \text{where $i=e, \mu, \tau$}
\end{equation}
which has the advantage of involving all neutrino flavors.

\subsection{Neutrino mixing and CP violation}
Several experiments which study the flux of neutrinos coming from a nuclear reactor, or of Solar and atmospheric neutrinos observed some discrepancies, such as,  for example, the fact that the flux of electron neutrinos arriving on Earth form the Sun is about an half of that expected from theory.
Neutrino oscillations i.e. electron neutrinos turning into muon and tau neutrinos, are able to explain these observations. However in this way conservation of Lepton number is violated, and what's more, neutrino mixing implies that $e$, $\mu$ and $\tau$ neutrinos are not massless, as conceived by Bruno Pontecorvo in 1968. %\cite?? (WRAP theses pag.8)
%say something on Majorana neutrinos??

\section{Theory}

\subsection{Dirac equation}
The Dirac equation was invented by Dirac in seeking to make Schrödinger's equation for an electron compatible with special relativity. The Dirac field $\psi$ is a four component spinor, which can describe both particles and antiparticles, with half-integer spin.
The Dirac equation is
\begin{equation}
\label{eq:dir}
(i\gamma^{\mu}\partial_{\mu}-m)\psi=0
\end{equation}
where the Dirac spinor is composed by a left-handed spinor and a right-handed spinor.
\begin{equation} 
\psi=\begin{pmatrix}\psi_1\\ \psi_2 \\ \psi_3 \\ \psi_4 \end{pmatrix}=\begin{pmatrix}\psi_L\\ \psi_R \end{pmatrix}
\end{equation}

From Dirac equation it is possible to build a Lagrangian density
\begin{equation}
\label{eq:DirLag}
\mathcal{L}=\bar{\psi}(i\gamma^{\mu}\partial_{\mu}-m)\psi
\end{equation}
where $\bar{\psi}$ is the row matrix $\bar{\psi}=\psi^{\dagger}\gamma^0$.

Equation (\ref{eq:dir}) has plane wave solutions of the form
\begin{equation}
\psi_L=u_Le^{i(\vec{p}\cdot\vec{r}-Et)},
\qquad
\psi_R=u_Re^{i(\vec{p}\cdot\vec{r}-Et)}
\end{equation}
where $u_L$ and $u_R$ are two component spinors. For simplicity lets choose the particle rest frame, i.e. in which $\vec{p}=0$. Since Dirac equation must satisfy special relativity we have that $E^2=p^2+m^2$ and therefore $E=\pm m$. We thus found positive and negative energy solutions%solve problem about spinors u and u_L
\begin{gather}
\psi_L=ue^{-imt},\qquad  \psi_R=ue^{-imt} \\
\psi_L=ve^{imt},\qquad  \psi_R=-ve^{imt}
\end{gather}
The latter have to be interpreted as positive energy antiparticle states.
The total Lagrangian density which describes weak interactions of leptons is determined by the requirement of $U_Y(1)\times SU_L(2)$ local gauge symmetry.
Generation of charged lepton masses is due to spontaneous symmetry breaking (SSB) after the coupling with the Higgs field.
This mechanism could in principle be applied also to neutrinos, however since values of neutrino masses are many order of magnitude smaller than those of other fermions it is very unlikely they have a common SM Higgs origin.
Another possibility is the method of the effective Lagrangian which, after SSB generates Majorana neutrinos with definite masses and suppression via seesaw mechanism of neutrino masses with respect to SM masses of leptons and quarks.

If neutrinos are Dirac particles total lepton number \emph{L} is conserved, and neutrino $\nu_i$ and antineutrino $\bar{\nu}_i$ have lepton number respectively  $L(\nu_i)=+1$ and $L(\bar{\nu}_i)=-1$.
For Majorana neutrinos instead L is not conserved and it's not possible to distinguish neutrinos from antineutrinos through any quantum number.

\subsection{Higgs mechanism for fermion masses}
Generation of fermion masses through the standard Higgs mechanism requires the introduction of a Yukawa-type coupling of the lepton SU(2) doublet to a scalar field, which then acquires a vacuum expectation value (vev): 
\begin{equation}
\label{eq:ElectronLag}
\mathcal{L}_Y^e(x)=-g_e(\bar{l}_{eL}\phi e_R+\bar{e}_R\phi l_{eL}) %aggiungi equazione con psi !!
\end{equation}
where
\begin{equation}
\psi_{lL}(x)=\begin{pmatrix}\nu_{e}(x)\\ e^-(x) \end{pmatrix}_L,\qquad
\phi=\begin{pmatrix}\phi_+(x)\\ \phi_0(x) \end{pmatrix}
\end{equation}
are lepton and Higgs doublets, while $e_R(x)$ is the right handed electron singlet field.
We spontaneously break the symmetry choosing 
\begin{equation}
\phi(x)=\begin{pmatrix}0\\ \frac{v+H(x)}{\sqrt{2}}\end{pmatrix}
\end{equation}
with $v=(\sqrt{2}G_F)^{-1/2}\simeq246GeV$ which is the vev of the Higgs field.
Lagrangian (\ref{eq:ElectronLag}) then becomes
\begin{equation}
\mathcal{L}_Y^e(x)=-\frac{g_e}{\sqrt{2}}v(\bar{e}_Le_R+\bar{e}_Re_L)-\frac{g_e}{\sqrt{2}}(\bar{e}_Le_R+\bar{e}_Re_L)H(x)
\end{equation}
Here we can make the identification $m_e=g_ev/\sqrt{2}$ which gives as the electron mass.

For neutrinos in principle the same mechanism is valid, so that we start from the Lagrangian
\begin{equation}
\label{eq:nuLag}
\mathcal{L}_Y^{\nu_e}(x)=-g_{\nu_e}(\bar{l}_{eL}\phi_C \nu_{eR}+\bar{\nu}_{eR}\phi_C l_{eL})
\end{equation}
where
\begin{equation}
\phi_C(x)=i\tau_2\phi*(x)=\begin{pmatrix}\bar{\phi}^0(x) \\ -\phi^-(x)\end{pmatrix}
\end{equation}
is the charge conjugated field. Through SSB we obtain $\phi_C(x)=\begin{pmatrix}(v+H(x))/\sqrt{2}\\0\end{pmatrix}$, such that Lagrangian (\ref{eq:nuLag}) can be rewritten as
\begin{equation}
\label{eq:nuMassLag}
\mathcal{L}_Y^{\nu_e}(x)=-\frac{g_{\nu_e}}{\sqrt{2}}v(\bar{\nu}_{eL}\nu_{eR}+\bar{\nu}_{eR}\nu_{eL})-\frac{g_{\nu_e}}{\sqrt{2}}(\bar{\nu}_{eL}\nu_{eR}+\bar{\nu}_{eR}\nu_{eL})H(x)
\end{equation}
We finally obtained a (Dirac) mass for the neutrino, setting $g_{\nu_e}v/\sqrt{2}=m_{\nu_e}$.

It is easy to check that the Lagrangian of the Standard Model with the Yukawa interaction~\ref{eq:nuMassLag} is invariant under the global transformation
\[
\nu_{eL}(x)\rightarrow e^{i\theta}\nu_{eL}(x),\quad\nu_{eR}(x)\rightarrow e^{i\theta}\nu_{eR}(x),
\quad e_L(x)\rightarrow e^{i\theta}e_L(x),\quad e_R(x)\rightarrow e^{i\theta}e_R(x)
\]
where $\theta$ is an arbitrary constant. Thus in the SM with massive neutrinos the total lepton number L is conserved  \emph{and $\nu_i(x)$ is the Dirac field of neutrinos (L=1) and antineutrinos (L=-1)}.

Although this mechanism of generation of neutrino masses is legitimate, if we look at the value of the Yukawa coupling $g_{\nu_e}$ it is many order of magnitudes smaller than the couplings for the other fermions:%cite ???
\[
2\cdot10^{-13}\leq g_{\nu_e}\leq10^{-12} \quad \text{while}\quad g_t\simeq7\cdot10^{-1}, g_b\simeq2\cdot10^{-2}, g_{\tau}\simeq7\cdot10^{-3}
\]
It looks extremely implausible that they have a common origin. This is also true if we think about smallness of neutrino masses, even if their absolute values are still unknown from neutrino oscillation and cosmological data the mass of the heaviest neutrino has been estimated to be
\[
5\cdot10^{-2}eV\simeq\biggl(\sqrt{\Delta m^2_A}\biggr)\leq m_3\leq\biggl(\frac{1}{3}\sum_im_i\biggr)\simeq3\cdot10^{-1}eV
\]
%\cite ???
where $\Delta m^2_A$ is the atmosphere neutrino mass-squared difference.

\subsection{Majorana neutrinos}
The Dirac mechanism for neutrino masses is not a completely exhaustive model which further requires the introduction of sterile %is it correct??
right handed neutrinos.
A new theory has been developed trying to build a more natural model for neutrino masses generation beyond the Standard Model. To be precise there is not a single theory but many. One of the most popular is the Weinberg effective Lagrangian: it allows the introduction of neutrino masses using only left-handed (Majorana) neutrino fields.
A consequence of this model is that the total lepton number L is not conserved, but apart from this all the results about neutrino oscillation and mixing coincide with those from SM predictions.

In Dirac massive particles there are four degrees of freedom ($\nu_{eL}$, $\nu_{eR}$ and their antiparticles), while for Majorana fermions they would be only two: $\nu_{eL}$ and $\nu_{eR}$. Furthermore, since RH neutrinos are $U(1)$ and $SU(2)$ singlets, they can be easily dropped altogether.
A Lorentz invariant mass term however requires a product of left-handed and right-handed components; this is achieved by the fact that
the charge conjugated field
\begin{equation}
\nu_{eL}^c=C\bar{\nu}_{eL}^T
\end{equation}
is right-handed. The Majorana mass term may be written as
\begin{equation}
-\frac{1}{2}m_L\bar{\nu}_{eL}\nu_{eL}^c+h.c.
\end{equation}
It is straightforward to see that the above mass term is not invariant under a global $U(1)$ transformation
\begin{equation}
\nu_{eL}\rightarrow e^{-i\theta}\nu_{eL}
\end{equation}
corresponding to non conservation  of lepton number L.
Considering all neutrino flavors the Majorana mass term has the form
\begin{equation}
\label{eq:MajLag}
\mathcal{L}^M=-\frac{1}{2}\sum_{l',l}\bar{\nu}_{l'L}M_{l',l}^L\nu_{lL}^c+h.c.
\end{equation}
where $M^M$ is a complex, $3\times3$ matrix. It can be diagonalized by the transformation $M^M=UmU^T$ with $UU^{\dagger}=1$ and $m_{ik}=m_i\delta_{ik},\quad m_i>0$.
Lagrangian (\ref{eq:MajLag}) thus becomes
\begin{equation}
\mathcal{L}^M=-\frac{1}{2}\sum_{i=1}^3m_i\bar{\nu}_i\nu_i
\end{equation}
where $\nu_i$ are mass eigenstates with eigenvalues $m_i$.
\begin{equation}
\nu_i=\sum_lU_{il}^{\dagger}\nu_{lL}+\sum_l\bigl(U_{il}^{\dagger}\nu_lL\bigr)^c
\end{equation}
The flavor neutrino fields $\nu_{lL}$ are given by a superposition of mass eigenstates $\nu_i$, whose contributions are weighted by coefficients $U_{il}$. The matrix \emph{U} is called PMNS matrix.

\subsection{Weinberg effective Lagrangian}
Since we are dealing with $\nu_{lL}$ field instead of the Dirac spinor $\nu_l$ it is not possible to generate Majorana masses through the usual coupling with Higgs field and the subsequent symmetry breaking. 
One way to proceed is through the $SU_L(2)\times U_Y(1)$ invariant
\begin{equation}
\label{eq:LorInv}
(\phi_C^{\dagger}\psi_{lL})\qquad(l=e, \mu, \tau)
\end{equation}
which after spontaneous symmetry breaking assumes the form
\begin{equation}
(\phi_C^{\dagger}\psi_{lL})_{SSB}=\frac{v}{\sqrt{2}}\nu'_{lL}
\end{equation}
From (\ref{eq:LorInv}) the only possible effective Lagrangian which generates neutrino masses is
\begin{equation}
\label{eq:EffLag}
\mathcal{L}_{eff}=\frac{1}{\Lambda}\sum_{l',l}(\phi_C^{\dagger}\psi_{l'L})X'_{l'l}(\phi_C^{\dagger}\psi_{lL})^c+h.c.
\end{equation}
where $X'$ is a symmetrical, non diagonal matrix.
The factor $1/\Lambda$ has been introduce since the mass dimension of (\ref{eq:EffLag}) is five while a Lagrangian should have mass dimension four, and indeed $\Lambda$ has dimension\emph{M}.
It is an important parameter since it can be interpreted as mass scale at which new physics appears.

After SSB Lagrangian (\ref{eq:EffLag}) the following neutrino mass term emerges
\begin{equation}
\label{eq:EffMass}
\mathcal{L}^M=-\frac{v^2}{2\Lambda}\sum_{l',l}\bar{\nu}'_{l'L}X'_{l'l}\nu_{lL}^{'c}+h.c.
\end{equation}
in which we can identify the Majorana matrix
\begin{equation}
M_L=\frac{v^2}{\Lambda}X. %is it M_L or M_2
\end{equation}
The matrix \emph{X} can be diagonalized by the unitary matrix \emph{U}: $X=UxU^T$. %look carefully to difference between X and X'
In this case Lagrangian (\ref{eq:EffMass}) becomes
\begin{equation}
\mathcal{L}^M=-\frac{1}{2}\sum_{i=1}^3m_{iL}\bar{\nu}_i\nu_i
\end{equation}
where
\begin{equation}
\label{eq:nuC}
\nu_i(x)=\nu_i^c(x)=C\bar{\nu}_i^T(x)
\end{equation}
is the field of the Majorana neutrino with mass 
\begin{equation}
\label{eq:neutMass}
m_{iL}=\frac{v}{\Lambda}(vx_i)
\end{equation}
and through a superposition can give neutrino field of definite flavor
\begin{equation}
\label{eq:nuMix}
\nu_{lL}=\sum_{i=1}^3U_{li}\nu_{iL}
\end{equation}
where \emph{U} is again the PMNS matrix.

From (\ref{eq:neutMass}) we see that neutrino masses are characterized by the two parameters \emph{v} and $\Lambda$; what's more it provides an explanation of the smallness of neutrino mass with respect to other fermions. The suppression is by a factor
\begin{equation}
\frac{v}{\Lambda}=\frac{\text{EW scale}}{\text{scale of a new physics}}
\end{equation}
which is naturally much smaller than one.
If we assume the mass of the heaviest neutrino (mass) field to be $m_3\simeq5\times10^{-2}eV$ and $x_3\leq1$ (although parameters $x_i$ are unknown) we obtain the estimate $\Lambda\leq10^{15}GeV$

\subsection{Origin of the Weinberg effective Lagrangian and the seesaw mechanism}
The Standard Model for three generations of neutrinos can be extended to contain three right-handed neutrinos $\nu_R$ building a gauge invariant renormalizable Lagrangian $\mathcal{L}=\mathcal{L}_{SM}+\mathcal{L}_{\nu_R}$ 
\begin{equation}
\label{eq:renormLag}
\mathcal{L}=i\bar{\nu}_L\sigma^{\mu}\partial_{\mu}\nu_L+i\bar{\nu}_R\sigma^{\mu}\partial_{\mu}\nu_R-\bar{\nu}_LM_D\nu_R-\frac{1}{2}\bar{\nu}^c_LM_L\nu_L-\frac{1}{2}\bar{\nu}_R^cM_R\nu_R+h.c.
\end{equation}
It contains a $3\times3$ Dirac mass matrix $M_D$ together with $3\times3$ Majorana mass matrices $M_L$ and $M_R$ for left-handed and right-handed neutrinos respectively. Notice that Majorana mass terms are not produced through Higgs mechanism and SSB. Working in the basis in which $M_R$ is real and diagonal, neutrino mass eigenstates are given by $\nu_{iR}=\nu_{iR}^c$. 
Writing the complete mass term for three neutrino generations results in 
\begin{equation}
\mathcal{L}_{mass}=-\frac{1}{2}\begin{pmatrix}\bar{\nu}_L \bar{\nu}_R^c\end{pmatrix}\begin{pmatrix}M_L & M_D\\ M_D^T & M_R\end{pmatrix}\begin{pmatrix}\nu_L^c\\ \nu_R\end{pmatrix}+h.c.
\end{equation}
where the mass matrix can be diagonalized by a $6\times6$ unitary transformation $U$ to give 
\begin{equation}
U^T\begin{pmatrix}M_L & M_D\\ M_D^T & M_R\end{pmatrix}U=\begin{pmatrix}M_1 & 0 \\ 0 & -M_2\end{pmatrix}
\end{equation}
$M_1$ and $M_2$ are $3\times3$ real diagonal matrices. In case of type-I seesaw mechanism the choice is $M_L=0$ and $M_D\ll M_R$.
Considering only one neutrino generation the total mass matrix becomes 
\begin{equation}
M=\begin{pmatrix}m_L & m_D\\ m_D^T & m_R\end{pmatrix}
\end{equation}
where $m_L, m_R$ and $m_D$ are all real parameters. Assuming $m_L=0$ and $m_D\ll m_R$ we get two Majorana masses %true???
\begin{equation}
m_1\approx m_R \qquad m_2\approx -m^2_D/m_R
\end{equation}
that are, one heavy right-handed neutrino of mass $m_1$ and a light left-handed neutrino of mass $m_2$. In this are obtained values for LH neutrino masses compatible with experimental bounds. Notice that this does not imply that $m_D$ has to be small compared to the electroweak scale, and thus the unnaturally small coupling $Y^{\nu}$ is avoided.

Values of neutrino masses we just derived can be obtained through an effective Lagrangian, valid at energies below the mass scale $M_R$. This Lagrangian is constructed by integrating out the heavy Majorana neutrino fields $\nu_{iR}$ and performing a power series expansion in $1/M_R$ of the form
\begin{equation}
\mathcal{L}_{eff}=\mathcal{L}_{SM}+\frac{1}{M_R}\mathcal{L}^{d=5}+\frac{1}{M_R^2}\mathcal{L}^{d=6}+\dots
\equiv \mathcal{L}_{SM}+\delta\mathcal{L}^{d=5}+\delta\mathcal{L}^{d=6}+\dots
\end{equation}
Here the $\delta\mathcal{L}^{d=5}$ effective Lagrangian corresponds to the Weinberg effective Lagrangian (\ref{eq:EffLag}) so that we can find the relation
\begin{equation}
\frac{1}{\Lambda}X'=Y^{*}_{\nu}\frac{1}{M_R}Y_{\nu}^{\dagger} %controlla correttezza di forma!!!
\end{equation}
This means that the scale of new physics $\Lambda$ is defined by the masses of heavy Majorana fields $\nu_{iR}$, and furthermore the mass matrix of light neutrinos is given by
\begin{equation}
M_2=-\frac{v^2}{2}\biggl(Y^{*}_{\nu}\frac{1}{M_R}Y_{\nu}^{\dagger}\biggr) %is it M_2 or M_L
\end{equation}


Interaction (\ref{eq:renormLag}) however is not the only way to generate the effective Weinberg Lagrangian. It can also derive from 
%Most neutrino Majorana masses can be parametrized by the Weinberg operator, but not always. If the models contain particles lighter than the electroweak scale, like light Dirac neutrinos or light sterile fermions.

\section{Nutrinoless double beta decay}
Double beta decay ($2\nu\beta\beta$) is the simultaneous beta decay of two neutrons in a nucleus: $(A, Z)\rightarrow(A, Z+2)+2e^-+2\bar{\nu}_e$.
It is a second order process and has been observed in even-even nuclei in which single beta decay is energetically forbidden. As we saw, some models that extend the SM predict the existence of a \emph{neutrinoless} double beta decay ($0\nu\beta\beta$)
\begin{equation}
(A, Z)\rightarrow(A, Z+2)+2e^-
\end{equation}
This process could be observed whether other possible decay channels are forbidden, and several ongoing experiments nowadays are trying to search for it. This will prove the existence of Majorana neutrinos.

\subsection{Theory}
The theory on $0\nu\beta\beta$-decay is based on the assumption that the CC interaction is the SM interaction
\begin{equation}
\mathcal{L}^{CC}=-\frac{g}{2\sqrt{2}}j_{\alpha}^{CC}W^{\alpha}+h.c.
\end{equation}
with
\begin{equation}
j_{\alpha}^{CC}=2\bar{\nu}_{lL}\gamma_{\alpha}l_L+j_{\alpha}^{CCquark}
\end{equation}
where $j_{\alpha}^{CCquark}$ is the quark charged current, $W^{\alpha}$ is the charged vector bosons field and $g$ is the constatnt of electroweak interaction. The flavor neutrino field $\nu_{lL}$ is given by the mixing relation (\ref{eq:nuMix}) of Majorana neutrinos (\ref{eq:nuC}).

The effective Hamiltonian of $\beta$-decay is 
\begin{equation}
\mathcal{H}=\frac{G_F}{\sqrt{2}}2\bar{e}_L\gamma_{\alpha}\nu_{eL}j^{\alpha}+h.c.
\end{equation}
The matrix element for neutrinoless double $\beta$-decay has to be calculated in the second order of the perturbation theory in $G_F$. Going directly to the result, the total decay rate for this process is
\begin{equation}
\Gamma^{0\nu}=\frac{ln(2)}{T^{0\nu}_{1/2}}=|m_{\beta\beta}|^2|M^{0\nu}|^2G^{0\nu}(Q,Z)
\end{equation}
where the matrix element is given by the product of the effective Majorana mass $m_{\beta\beta}$ and the nuclear matrix element $M^{0\nu}$, and $G^{0\nu}(Q,Z)$ is the phase space factor, which includes effects of Coulomb interaction between electrons and the daughter nucleus. For a complete calculation of the decay rate see~\cite{ref:Bil}.

A particular focus as to be set on the effective Majorana mass
\begin{equation}
m_{\beta\beta}=sum_{i=1}^3U_{ek}^2m_k
\end{equation}
since it depends on neutrino masses (still unknown) and on the matrix elements of the PMNS matrix, which are extracted from neutrino oscillation experiments.
From studying solar neutrinos we get $\Delta m_S^2\equiv\Delta m_{12}^2>0$, while from atmospheric neutrino experiments $\Delta m_A^2$ is derived.
At this point however there a re two possibilities:
\begin{enumerate}
\item Normal ordering (NO)
\begin{equation}
m_3>m_2>m_1
\end{equation}
\item Inverted ordering (IO)
\begin{equation}
m_2>m_1>m_3
\end{equation}
\end{enumerate}
In the first case (NO) $\Delta m_A^2\equiv\Delta m_{23}^2$ and
\begin{equation}
m_2=\sqrt{m_1^2+\Delta m_S^2} \qquad m_3=\sqrt{m_1^2+\Delta m_S^2+\Delta m_A^2}
\end{equation}
while for inverted ordering $\Delta m_A^2\equiv\Delta m_{13}^2$ and
\begin{equation}
m_1=\sqrt{m_3^2+\Delta m_A^2} \qquad m_2=\sqrt{m_3^2+\Delta m_S^2+\Delta m_A^2}
\end{equation}

For what concerns the PMNS matrix, if we are dealing with Dirac neutrinos it is characterized by three mixing angles $\theta_{12}, \theta_{23}, \theta_{13}$ and one phase $\delta$. In the standard parametrization it has the form
\begin{equation}
U^D=\begin{pmatrix}c_{13}c_{12}&c_{13}s_{12}&s_{13}e^{-i\delta}\\-c_{23}s_{12}-s_{23}c_{12}s_{13}e^{i\delta}&c_{23}c_{12}-s_{23}s_{12}s_{13}e^{i\delta}&c_{13}s_{23}\\s_{23}s_{12}-c_{23}c_{12}s_{13}e^{i\delta}&-s_{23}c_{12}-c_{23}s_{12}s_{13}e^{i\delta}&c_{13}c_{23}\end{pmatrix}
\end{equation}
where $c_{ij}\equiv cos(\theta_{ij})$ and $s_{ij}\equiv sin(\theta_{ij})$.

If instead neutrinos are Majorana particles, two additional phases $\alpha_1$ and $\alpha_2$ enter in the mixing matrix
\begin{equation}
U^M=U^DS^M(\alpha)
\end{equation}
where
\begin{equation}
s^M(\alpha)=\begin{pmatrix}e^{i\alpha_1}&0&0\\0&e^{i\alpha_2}&0\\0&0&1\end{pmatrix}
\end{equation}



\subsection{Experiments}




\subsection{Mathematics}
Here is a lettered array~(\ref{e.all}), with eqs.~(\ref{e.house})
and~(\ref{e.phi}):
\begin{eqnletter}
 \label{e.all}
 \drm x_\sy{F} & = & 1.2\cdot10^3\un{cm}, \qquad
                     \tx{where\ } \sy{F} = \tx{Fermi}    \label{e.house}\\
 \phi_i        & = & i\pi                                \label{e.phi}
\end{eqnletter}

\subsection{Citations}
We're almost done, just some citations~\cite{ref:apo}
and we will be over~\cite{ref:pul,ref:bra}.


\appendix

\section{}
Let us go then, you and I\ldots

\acknowledgments
This work was produced, supported and perpetrated by M. Bellacosa under
the auspices of the Italian Physical Society.

\begin{thebibliography}{0}
\bibitem{ref:apo} \BY{Boccaccio~G. \atque de~Cam\~oes~L.}
  \IN{Phys. Rev. A}{13}{1999}{12};
  \SAME{69}{999}{1666}.
\bibitem{ref:pul} \BY{Pulci~L.}
  preprint INFN 8181.
\bibitem{ref:bra} \BY{Bragg~B.}
  \TITLE{Tender comrade},
  in \TITLE{Workers Playtime},
                  edited by \NAME{Tizio A. \atque Caio B.}
                  (Unexeditor, Bologna) 1997, pp.~1-10.

\bibitem{ref:cott} \BY{Cottingham~W.N. \atque Greenwood~D.A.} 
\TITLE {An introduction to the Standard Model of Particle Physics. Second edition.} Cambridge University Press, 2007.
\bibitem{ref:ait2} \BY{Aitchison~I.J.R. \atque Hey~A.J.G.} \TITLE{Gauge Theories in Particle Physics. A practical introduction. Vol.2}, in \TITLE{Graduate student series in Physics}, edited by \NAME{Brewer~D.A.} (Institute of Physics Publishing, Bristol and Philadelphia) 2004.
%pp368-372
\bibitem{ref:Halzen} \BY{Halzen~F. \atque Martin~A.D.} \TITLE{Quarks and Leptons: An Introductory Course in Modern Particle Physics} 
(John Wiley \& Sons.) 1984. %pp 338-340
\bibitem{ref:Bronc} \BY{Broncano~A. , Gavela~M.B. \atque Jenkins~E.} \TITLE{The Effective Lagrangian for the Seesaw Model of Neutrino Mass and Leptogenesis}, (2006)  [hep-ph/0210271].
\bibitem{ref:Juji} \BY{Fujikawa~K.} \IN{Eur. Phys. J. C}{80}{2020}{285}.
\bibitem{ref:Bil} \BY{Bilenky~S.M.} \TITLE{Neutrinos: Majorana or Dirac?}, (2020) [arXiv:2008.02110v1].

\end{thebibliography}

\end{document}